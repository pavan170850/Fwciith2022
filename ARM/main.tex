\def\mytitle{ BOOLEAN LOGIC USING VAMAN }
\def\myauthor{K.Pavan Kumar}
\def\contact{r170850@rguktrkv.ac.in}
\def\mymodule{Future Wireless Communications (FWC)}
\documentclass[10pt, a4paper]{article}
\usepackage[a4paper,outer=1.5cm,inner=1.5cm,top=1.75cm,bottom=1.5cm]{geometry}
\twocolumn
\usepackage{graphicx}
\graphicspath{{./images/}}
\usepackage[colorlinks,linkcolor={black},citecolor={blue!80!black},urlcolor={blue!80!black}]{hyperref}
\usepackage[parfill]{parskip}
\usepackage{lmodern}
\usepackage{tikz}

\usepackage{karnaugh-map}

\usepackage{tabularx}
%\documentclass{article}
%\documentclass[tikz, border=2mm]{standalone}
\usepackage{tikz}
\usepackage{circuitikz}
\usetikzlibrary{calc}

\renewcommand*\familydefault{\sfdefault}
\usepackage{watermark}
\usepackage{lipsum}
\usepackage{xcolor}
\usepackage{listings}
\usepackage{float}
\usepackage{titlesec}

\titlespacing{\subsection}{1pt}{\parskip}{3pt}
\titlespacing{\subsubsection}{0pt}{\parskip}{-\parskip}
\titlespacing{\paragraph}{0pt}{\parskip}{\parskip}
\newcommand{\figuremacro}[5]{
    \begin{figure}[#1]
        \centering
        \includegraphics[width=#5\columnwidth]{#2}
        \caption[#3]{\textbf{#3}#4}
        \label{fig:#2}
    \end{figure}
}

\lstset{
frame=single, 
breaklines=true,
columns=fullflexible
}

\thiswatermark{\centering \put(0,-110.0){\includegraphics[scale=0.3]{logo}} }
\title{\mytitle}
\author{\myauthor\hspace{1em}\\\contact\\FWC22011\hspace{6.5em}IITH\hspace{0.5em}\mymodule\hspace{5em}Assignment:ARM}
\date{}
\begin{document}
  \maketitle
  \tableofcontents
  \begin{abstract}
      This manual explains logic Circuit for the following Boolean Expression using only NOR Gates  :
      
      \begin{center}
      (A+B).(C+D)
      \end{center}

  \end{abstract}
\section{Introduction}
  \begin{tabularx}{0.4\textwidth} { 
  | >{\centering\arraybackslash}X 
  | >{\centering\arraybackslash}X 
  | >{\centering\arraybackslash}X | }
\hline
 \textbf{A}& \textbf{B} & \textbf{Y}\\
\hline
0 & 0 & 1 \\  
\hline
0&1&0 \\ 
\hline
1&0&0\\
\hline
1&1&0\\
\hline
\end{tabularx}
\begin{center}
Truth Table for NOR Gate
\end{center}

    \subsection{NOR Gate:}
Use  two input NOR Gate.The truth table for  NOR gate is shown above.In the truth table above A,B are inputs and Y is the output.




  \section{Components}
  \begin{tabularx}{0.4\textwidth} { 
  | >{\centering\arraybackslash}X 
  | >{\centering\arraybackslash}X 
  | >{\centering\arraybackslash}X | }
\hline
 \textbf{Component}& \textbf{Values} & \textbf{Quantity}\\
\hline
VAMAN Board &  & 1 \\  
\hline
JumperWires& M-F & 7 \\ 
\hline
Breadboard &  & 1 \\
\hline
USB-B cable &  & 1 \\
\hline
\end{tabularx}

\section{Circuit Diagram}
\begin{circuitikz} \draw

(0,2) node[nor port]  (mynor1) {}
(0,0) node[nor port] (mynor2) {}
(2,1) node[nor port] (mynor) {}
(mynor1.out) -- (mynor.in 1)
(mynor2.out) -- (mynor.in 2);

\node[left] at (mynor1.in 1) {\(A\)};
\node[left] at (mynor1.in 2) {\(B\)};

\node[left] at (mynor2.in 1) {\(C\)};
\node[left] at (mynor2.in 2) {\(D\)};
\node[right] at (mynor1.out) {\((A+B)'\)};
\node[right] at (mynor2.out) {\((C+D)'\)};

\node[right] at (mynor.out) {\(Y=(A+B).(C+D)\)};
\end{circuitikz}

\begin{center}
Figure.a
\end{center}
\paragraph{Evaluation of Output  Y :}
In Digital Electronics the De Morgan's Laws are :
\\First Law: (AB)' = A'+B'.
\\Second Law: (A+B)' = A'.B'.
\\The output Y=((A+B)'+(C+D)')'
\\ =((A+B)')'.((C+D)')'     (From De Morgan's Second Law)
\\ =(A+B).(C+D)             since ((A)')'=A


\section{Truth Table}
  \begin{tabularx}{0.46\textwidth} { 
  | >{\centering\arraybackslash}X 
  | >{\centering\arraybackslash}X 
  | >{\centering\arraybackslash}X
  | >{\centering\arraybackslash}X 
  | >{\centering\arraybackslash}X | }


\hline
 D & C & B & A  & Y\\
\hline
0 & 0 & 0 & 0 & 0 \\  
\hline
0 & 0 & 0 & 1 & 0 \\ 
\hline
0 & 0 & 1 & 0 & 0 \\
\hline
0 & 0 & 1 & 1 & 0 \\
\hline
0 & 1 & 0 & 0 & 0 \\  
\hline
0 & 1 & 0 & 1 & 1\\ 
\hline
0 & 1 & 1 & 0 & 1 \\
\hline
0 & 1 & 1 & 1 & 1 \\
\hline
1 & 0 & 0 & 0 & 0 \\
\hline
1 & 0 & 0 & 1 & 1 \\
\hline
1 & 0 & 1 & 0 & 1 \\
\hline
1 & 0 & 1 & 1 & 1 \\
\hline
1 & 1 & 0 & 0 & 0 \\
\hline
1 & 1 & 0 & 1 & 1 \\
\hline
1 & 1 & 1 & 0 & 1 \\
\hline
1 & 1 & 1 & 1 & 1 \\
\hline
\end{tabularx}
\begin{center}
TABLE 1
\end{center}
\begin{center}

\section{K-map}
     \begin{karnaugh-map}[4][4][1][$BA$][$DC$]
        \minterms{5,6,7,9,10,11,13,14,15}
        \maxterms{0,1,2,3,4,8,12}
        \implicant{5}{15}
        \implicant{7}{14}
        \implicant{13}{11}
        \implicant{15}{10}
    \end{karnaugh-map}
\end{center}
\begin{center}
Figure.b
\end{center}
    \paragraph{Karnugh Map for Y:}
 
Draw k-map for  the truth table shown in Table 1 .The Given expression  (A+B).(C+D) is obtained by  using the K-map in Fig.b 
The implicants in boxes 5,7,13,15 result in "AC" ,the implicants in boxes 9,11,13,15 result in "AD",the implicants in boxes 6,11,14,15 result in "BC",the implicants in boxes 10,11,14,15 result in "BD",
\\The output is expressed in terms of inputs D,C,B,A as: 
\\Y=AC+BC+AD+BD
\\ =C(A+B)+D(A+B)
\\ =(A+B).(C+D)


The code below realizes the Boolean logic for Y , using 5V,GND of Vaman Board .
\\
2,4,6,8 GPIO Pins of Vaman Board are configured as input pins and the required Logic for D,C,B,A are drawn from 5V (Logic '1'),GND (Logic '0'). Built in led at 22nd pin will glow based on Y satisfying the Table-1
\begin{center}
\fbox{\parbox{8.5cm}{\url{https://github.com/pavan170850/FWC_module1/blob/main/arm/codes/src/main.c}}}
\end{center}
\section{Setup}
\begin{enumerate}
\item Connect the Vaman to the Laptop through USB.
\item There is a button and an LED to the left of the USB port on the Vaman.There is another button to the right of the LED.
\item Press the right button first and immediately press the left button.The LED will be blinking green.The Vaman is now in bootloader mode.
\end{enumerate}
\subsection{The steps for implementation:}
\begin{enumerate}
\item Login to termux-ubuntu on the android device and execute the following commands:\\
Make sure that the required installation of pygmy-sdk had done prior executing below commands
\begin{lstlisting}
proot-distro login debian
cd  /data/data/com.termux/files/home/
mkdir arm
cd arm
svn co https://github.com/pavan170850/Fwciith2022/trunk/ARM/codes
\end{lstlisting}
\begin{lstlisting}
cd codes/GCC_Project
make -j4
scp /data/data/com.termux/files/home/arm/codes/GCC_Project/output/bin/codes.bin usernameofpc@IPaddress:/home/username
\end{lstlisting}
Make sure that the appropriate username,IP address of the Laptop is given in the above command.
\item Now execute the following commands on the Laptop terminal\\
Make sure that required installation of programmer application and modification of bash file had done prior executing below command
\begin{lstlisting}
bash flash.sh codes.bin
\end{lstlisting}
\item After finishing the process of flashing with the programmer application press the button to the right of the USB port to reset. Vaman is now flashed with our source code
\end{enumerate}
\textbf{Note:}
while executing the code using "make -j4 ",execution will be faster as compared when executing with "make" command.The reason is "make -j4" uses four cores while make uses only one core to process the code.


\bibliographystyle{ieeetr}
\end{document}