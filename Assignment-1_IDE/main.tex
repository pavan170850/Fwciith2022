\def\mytitle{IMPLEMENTATION OF BOOLEAN LOGIC IN ARDUINO IDE}
\def\myauthor{K.Pavan Kumar}
\def\contact{r170850@rguktrkv.ac.in}
\def\mymodule{Future Wireless Communications (FWC)}
\documentclass[10pt, a4paper]{article}
\usepackage[a4paper,outer=1.5cm,inner=1.5cm,top=1.75cm,bottom=1.5cm]{geometry}
\twocolumn
\usepackage{graphicx}
\graphicspath{{./images/}}
\usepackage[colorlinks,linkcolor={black},citecolor={blue!80!black},urlcolor={blue!80!black}]{hyperref}
\usepackage[parfill]{parskip}
\usepackage{lmodern}
\usepackage{tikz}

\usepackage{karnaugh-map}

\usepackage{tabularx}
%\documentclass{article}
%\documentclass[tikz, border=2mm]{standalone}
\usepackage{tikz}
\usepackage{circuitikz}
\usetikzlibrary{calc}

\renewcommand*\familydefault{\sfdefault}
\usepackage{watermark}
\usepackage{lipsum}
\usepackage{xcolor}
\usepackage{listings}
\usepackage{float}
\usepackage{titlesec}

\titlespacing{\subsection}{1pt}{\parskip}{3pt}
\titlespacing{\subsubsection}{0pt}{\parskip}{-\parskip}
\titlespacing{\paragraph}{0pt}{\parskip}{\parskip}
\newcommand{\figuremacro}[5]{
    \begin{figure}[#1]
        \centering
        \includegraphics[width=#5\columnwidth]{#2}
        \caption[#3]{\textbf{#3}#4}
        \label{fig:#2}
    \end{figure}
}

\lstset{
frame=single, 
breaklines=true,
columns=fullflexible
}

\thiswatermark{\centering \put(-15,-100.0){\includegraphics[scale=0.3]{logo}} }
\title{\mytitle}
\author{\myauthor\hspace{1em}\\\contact\\FWC22011\hspace{6.5em}IITH\hspace{0.5em}\mymodule\hspace{5em}ASSIGNMENT-1}
\date{}
\begin{document}
  \maketitle
  \tableofcontents
  \begin{abstract}
      This manual explains logic Circuit for the following Boolean Expression using only NOR Gates  :
      
      \begin{center}
      (A+B).(C+D)
      \end{center}

  \end{abstract}
\section{Introduction}
  \begin{tabularx}{0.4\textwidth} { 
  | >{\centering\arraybackslash}X 
  | >{\centering\arraybackslash}X 
  | >{\centering\arraybackslash}X | }
\hline
 \textbf{A}& \textbf{B} & \textbf{Y}\\
\hline
0 & 0 & 1 \\  
\hline
0&1&0 \\ 
\hline
1&0&0\\
\hline
1&1&0\\
\hline
\end{tabularx}
\begin{center}
Truth Table for NOR Gate
\end{center}

    \subsection{NOR Gate:}
Use  two input NOR Gate.The truth table for  NOR gate is shown above.In the truth table above A,B are inputs and Y is the output.

\subsection{Arduino:}
    The Arduino Uno has some ground pins, analog
    input pins A0-A3 and digital pins D1-D13 that can be used for both input as well as output. It also has two power pins that can generate 3.3V and 5V.Here I had used  only the GND, 5V and digital pins.


  \section{Components}
  \begin{tabularx}{0.4\textwidth} { 
  | >{\centering\arraybackslash}X 
  | >{\centering\arraybackslash}X 
  | >{\centering\arraybackslash}X | }
\hline
 \textbf{Component}& \textbf{Values} & \textbf{Quantity}\\
\hline
Arduino & UNO & 1 \\  
\hline
JumperWires& M-M & 6 \\ 
\hline
Breadboard &  & 1 \\
\hline
\end{tabularx}

\section{Circuit Diagram}
\begin{circuitikz} \draw

(0,2) node[nor port]  (mynor1) {}
(0,0) node[nor port] (mynor2) {}
(2,1) node[nor port] (mynor) {}
(mynor1.out) -- (mynor.in 1)
(mynor2.out) -- (mynor.in 2);

\node[left] at (mynor1.in 1) {\(A\)};
\node[left] at (mynor1.in 2) {\(B\)};

\node[left] at (mynor2.in 1) {\(C\)};
\node[left] at (mynor2.in 2) {\(D\)};
\node[right] at (mynor1.out) {\((A+B)'\)};
\node[right] at (mynor2.out) {\((C+D)'\)};

\node[right] at (mynor.out) {\(Y=(A+B).(C+D)\)};
\end{circuitikz}

\begin{center}
Figure.a
\end{center}
\paragraph{Evaluation of Output  Y :}
In Digital Electronics the De Morgan's Laws are :
\\First Law: (AB)' = A'+B'.
\\Second Law: (A+B)' = A'.B'.
\\The output Y=((A+B)'+(C+D)')'
\\ =((A+B)')'.((C+D)')'     (From De Morgan's Second Law)
\\ =(A+B).(C+D)             since ((A)')'=A


\section{Truth Table}
  \begin{tabularx}{0.46\textwidth} { 
  | >{\centering\arraybackslash}X 
  | >{\centering\arraybackslash}X 
  | >{\centering\arraybackslash}X
  | >{\centering\arraybackslash}X 
  | >{\centering\arraybackslash}X | }


\hline
 D & C & B & A  & Y\\
\hline
0 & 0 & 0 & 0 & 0 \\  
\hline
0 & 0 & 0 & 1 & 0 \\ 
\hline
0 & 0 & 1 & 0 & 0 \\
\hline
0 & 0 & 1 & 1 & 0 \\
\hline
0 & 1 & 0 & 0 & 0 \\  
\hline
0 & 1 & 0 & 1 & 1\\ 
\hline
0 & 1 & 1 & 0 & 1 \\
\hline
0 & 1 & 1 & 1 & 1 \\
\hline
1 & 0 & 0 & 0 & 0 \\
\hline
1 & 0 & 0 & 1 & 1 \\
\hline
1 & 0 & 1 & 0 & 1 \\
\hline
1 & 0 & 1 & 1 & 1 \\
\hline
1 & 1 & 0 & 0 & 0 \\
\hline
1 & 1 & 0 & 1 & 1 \\
\hline
1 & 1 & 1 & 0 & 1 \\
\hline
1 & 1 & 1 & 1 & 1 \\
\hline
\end{tabularx}
\begin{center}
TABLE 1
\end{center}
\begin{center}

\section{K-map}
     \begin{karnaugh-map}[4][4][1][$BA$][$DC$]
        \minterms{5,6,7,9,10,11,13,14,15}
        \maxterms{0,1,2,3,4,8,12}
        \implicant{5}{15}
        \implicant{7}{14}
        \implicant{13}{11}
        \implicant{15}{10}
    \end{karnaugh-map}
\end{center}
\begin{center}
Figure.b
\end{center}
    \paragraph{Karnugh Map for Y:}
 
Draw k-map for  the truth table shown in Table 1 .The Given expression  (A+B).(C+D) is obtained by  using the K-map in Fig.b 
The implicants in boxes 5,7,13,15 result in "AC" ,the implicants in boxes 9,11,13,15 result in "AD",the implicants in boxes 6,11,14,15 result in "BC",the implicants in boxes 10,11,14,15 result in "BD",
\\The output is expressed in terms of inputs D,C,B,A as: 
\\Y=AC+BC+AD+BD
\\ =C(A+B)+D(A+B)
\\ =(A+B).(C+D)

 \section{Implementation}
     \subsection{Method 1}
Connect Arduino Uno to  computer.Built in led at 13th pin of Arduino will glow if the  output Y is logic'1',and off when output Y is logic'0' .

Observe the output by changing the input values from 0000 to 1111 in binary and verify the truth table.
\\The code below realizes the Boolean logic for Output Y
\begin{lstlisting}
https://github.com/pavan170850/Fwciith2022/blob/main/Assignment-1_IDE/codes/Method1/src/main.cpp
\end{lstlisting}


\subsection{METHOD-2:Using if else statement}
 Connect Arduino uno to  computer .
Built in led at 13th pin of Arduino will glow if the  output Y is logic'1',and off when output Y is logic'0' 

Observe the output by changing the input values from 0000 to 1111 in binary and verify the truth table.
\\The code below realizes the Boolean logic for Output Y using if else statement.

\begin{lstlisting}
https://github.com/pavan170850/Fwciith2022/blob/main/Assignment-1_IDE/codes/Method2/src/main.cpp
\end{lstlisting}

\subsection{METHOD-3:By giving  Manual inputs}
Make connections from Arduino Uno to Breadboard as shown in Table 2.
'0' means giving  0V or Logic LOW and '1' means giving logic HIGH or 5V.
\\Built in led at 13th pin is turned off ,Because the output Y have logic '0'.

  \begin{tabularx}{0.4\textwidth} { 
  | >{\centering\arraybackslash}X 
  | >{\centering\arraybackslash}X  | }
\hline
 \textbf{Arduino Digital Pins}& \textbf{Breadboard} \\
\hline
DP2 & 0 \\  
\hline
DP3 &0  \\ 
\hline
DP4 & 0 \\
\hline
DP5 & 0 \\
\hline
\end{tabularx}

\begin{center}
Table 2
\end{center}

Observe the output by changing the input values from 0000 to 1111 in binary and verify the truth table
\\The code below realizes the Boolean logic for Output Y
\begin{lstlisting}
https://github.com/pavan170850/Fwciith2022/blob/main/Assignment-1_IDE/codes/Method3/src/main.cpp
\end{lstlisting}

\bibliographystyle{ieeetr}
\end{document}
